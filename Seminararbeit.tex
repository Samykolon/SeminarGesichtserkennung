\documentclass[doktyp=semarbeit, sprache=german]{TUBAFarbeiten}
\usepackage[utf8]{inputenc}
\usepackage[T1]{fontenc}
\usepackage{graphicx} 
\usepackage{amsmath}
\usepackage{subcaption}
\usepackage{booktabs}
\usepackage{url}
\captionsetup{compatibility=false}
\bibliographystyle{unsrt}
\TUBAFFakultaet{Fakultät für Maschinenbau, Verfahrens- und Energietechnik}
\TUBAFInstitut{Institut für Automatisierungstechnik}
\TUBAFTitel[Moderne Methoden der Bildverarbeitung am Beispiel Gesichtserkennung]{Moderne Methoden der Bildverarbeitung am Beispiel Gesichtserkennung}
\TUBAFUntertitel{Anwendung von Informations- und Automatisierungssystemen}
\TUBAFBetreuer{Prof. Andreas Rehkopf}
\TUBAFAutor[S. Dressel und H. Krumbiegel]{Samuel Dressel und Hannes Krumbiegel}
\TUBAFDatum{\today}
\begin{document}
\maketitle
\tableofcontents
\newpage
\section{Einleitung}
\section{Grundlagen}
Einführend zum besseren Verständnis des Themenkomplexes ist es wichtig, einige wesentliche Begriffe zu definieren und außerdem voneinander abzugrenzen. Zunächst gilt es, zwischen der Lokalisation eines Gesichtes an sich und der Zuordnung des Gesichts zu einer bestimmten Person zu unterscheiden. Im ersten Fall wird geprüft, ob und wo ein Gesicht zu sehen ist, im zweiten, um wen es sich handelt \cite{FaceRecognitionWikipedia}. Wir sprechen somit zum einen von der Gesichtserkennung und zum anderen von der Gesichtsidentifizierung. Im englischen Sprachgebrauch unterteilt man den Teil der Gesichtsidentifizierung zusätzlich noch in eine menschliche Erkennung (\textit{face perception}) und in eine maschinelle Erkennung (\textit{face recognition}). In dieser Arbeit wird sich jedoch lediglich auf die maschinelle Gesichtsidentifizierung beschränkt.
\section{Gesichtserkennung (Face Detection)}
\section{Gesichtsidentifizierung (Face Recognition)}
\section{Zusammenfassung und Ausblick}
\bibliography{literatur}{}
\addcontentsline{toc}{section}{Literatur} 
\end{document}